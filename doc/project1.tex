\documentclass[11pt]{article}

\usepackage[utf8]{inputenc}
\usepackage{amsmath}
\usepackage{mathtools}
\usepackage{amsfonts}
% For figures and graphics'n stuff
\usepackage{graphicx}
\usepackage{caption}
\usepackage{subcaption}
\usepackage{url}
\usepackage{color}
\usepackage{float}
\usepackage{hyperref}
\usepackage[toc,page]{appendix}
\usepackage{enumerate}

% Correct way for typing C++
\newcommand{\CC}{C\nolinebreak\hspace{-.05em}\raisebox{.4ex}{\tiny\bf +}\nolinebreak\hspace{-.10em}\raisebox{.4ex}{\tiny\bf +}}
\def\CC{{C\nolinebreak[4]\hspace{-.05em}\raisebox{.4ex}{\tiny\bf ++}}}

% For cmd line arguments
\usepackage{listings}

\lstset{basicstyle=\footnotesize\ttfamily,breaklines=true}
\lstset{framextopmargin=50pt}


\lstdefinestyle{custom-pro-file}{
  % belowcaptionskip=1\baselineskip,
  % frame=L,
  % xleftmargin=\parindent,
  % language=C++,
  basicstyle=\footnotesize\ttfamily,
  % commentstyle=\itshape\color{green!40!black},
  % keywordstyle=\bfseries\color{purple!40!black},
  identifierstyle=\color{black},
}


% For proper citations
% \usepackage[round, authoryear]{natbib}
\usepackage[numbers]{natbib} 

% For color
\hypersetup{colorlinks=true,linkcolor=blue, linktocpage}

% For fixing large table height
\usepackage{a4wide}

\title{Machine Learning and Terrain data}
\author{Mathias M. Vege}

\date{\today}
\begin{document}
\maketitle

\begin{abstract}
None
\end{abstract}

\tableofcontents

\section{Introduction}
A now-fully emergent field of data analysis, is that of regression and resampling. Included as a subset of the field of machine learning, regression is widely used as tool of prediction. Together with resampling we are offered ways of estimating the error of our models.

One common way to investigate the efficacy of a model is to use the Franke function\cite{franke1979critical}.

\section{Methods and theory}
\subsection{Linear regression}
Linear regression is a method of performing a fit of parameters $x_i$ to a given data set $y_i$. Linear is a misnomer in the sense that linear does not mean a linear equation, but as linear in the fit coefficients, $\hat{\beta}$. The simplest case of linear regression is just a simple line\citep[ch. 3.1, p. 61]{james2013introduction}. This can be generalized greatly, and we end up with what is known as \textit{Ordinary Least Squares}.

\subsubsection{Ordinary Least Squares regression(OLS)}
To understand ordinary least squares, we can imagine ourselves wishing to fit some data $x_i$ to a desired output $y_i$ by some polynomial function. The data $y_i$ can have been generated by a function $f$. We begin by writing an equation in which we are trying to make a linear combination of $\hat{\beta}$ values with some $x_i$ to make an approximate $\tilde{y}_i$.

\begin{align*}
    \tilde{y}_i = x_{i,0}\beta^0 + 
\end{align*}

\subsubsection{Ridge regression}
\subsubsection{Lasso regression}

\subsection{Mean Square Error(MSE)}
The mean square error, popularly called MSE is a function that 
\subsection{\texorpdfstring{$R^2$}{R1} score}
\subsection{Bootstrapping}
\subsection{\texorpdfstring{$k$}{k}-fold Cross Validation}

\section{Implementation}
\section{Results}
\section{Discussion and conclusion}
\cite{scikit-learn}

\section{Appendix} \label{sec:appendix}
\subsection{The Franke Function}

\bibliography{bib/scikit.bib}
\bibliographystyle{plain}


\end{document}